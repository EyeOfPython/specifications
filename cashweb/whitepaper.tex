\documentclass{article}

\usepackage[superscript,biblabel]{cite}
\usepackage{savesym}
\usepackage{arxiv}
\usepackage{amsthm}
\usepackage{url}
\usepackage{graphics}

\usepackage[utf8]{inputenc} % allow utf-8 input
\usepackage[T1]{fontenc}  % use 8-bit T1 fonts
\usepackage{hyperref}    % hyperlinks
\usepackage{url}      % simple URL typesetting
\usepackage{booktabs}    % professional-quality tables
\usepackage{amsfonts}    % blackboard math symbols
\usepackage{nicefrac}    % compact symbols for 1/2, etc.
\usepackage{microtype}   % microtypography
\usepackage{lipsum}
\usepackage{float}
\usepackage{msc}

\drawframe{no}
\setlength{\instdist}{5cm}
\setlength{\instwidth}{2.5cm}
\setlength{\envinstdist}{3cm}
\setlength{\actionwidth}{3.5cm}

\title{CashWeb}

\author{
 Shammah Chancellor \\
 \texttt{shammah.chancellor@cashweb.io} \\
 \And
 Harry Barber \\
 \texttt{harry.barber@cashweb.io} \\
}

\begin{document}
\maketitle

\begin{abstract}
CashWeb is a suite of protocols providing an open messaging standard with a reference implementation called ``Stamp''. It uses a federated network topology, similar to XMPP and SMTP, to provide its scaling characteristics. It is unique in that it provides end-to-end encryption and abuse mitigation without the need for central moderation.
\end{abstract}

\section{Introduction}

\subsection{History}

Many early adopters of the internet saw it as a platform for inexpensive and fast sharing of new ideas. Systems like Usenet\cite{rfc5536}\cite{rfc5537}, Email\cite{rfc5322}\cite{rfc1939}\cite{rfc5321}\cite{rfc4551}, and XMPP\cite{rfc3920}\cite{rfc3921}\cite{rfc3922}\cite{rfc3923} were envisioned as decentralized platforms for this purpose.

However, due to the economics of network messaging the cost of message processing is placed on the recipient with respect to processing power, storage, and mental attention. We are increasingly relying on centralized systems to identify and filter these messages. Users delegated authority over their online identities and communications to third-parties in exchange for convenience. As a result, a loss of user privacy and control over online identities has occured.

As of June 2017, Facebook reported 2 billion users. By October 2018, GMail reported 1.5 billion active users\cite{gmail2018} and Cloudflare routed 10\% all internet traffic\cite{cloudflare2018}, June 2020, Google, Apple, and Microsoft held a total of 85\% of total market share of email clients\cite{emailshare2020}. We now rely on very few companies to reliably, and honestly, manage critical internet services.

The centralization of the internet into a few large platforms, while providing the benefit of a seamless experience across the web, resulted from the inability to construct decentralized alternatives with similar convenience. However, since the advent of Bitcoin, it is possible to construct decentralized systems which do provide these conveniences. The CashWeb protocols are centered around leveraging cryptocurrency to provide seemless online experiences -- thus restoring user authority over their digital identities and privacy. 

\subsection{Forces of Centralization}

\subsubsection{Identity Management}

In the past, internet service providers (ISPs) offered email services to their customers. This meant that when a user moved residences or wanted to switch service providers, their email address would change (e.g. john.doe@sonic.net). Using an email accounts provided by global companies like Google, and Microsoft, enabled users to avoid the effort associated with updating their contacts -- and potentially missing valuable communications.

Due to the delegation of address management, our digital identities have become more stable. As a result, businesses and users have begun to rely on them for more and more important communication and as a digital identity. Now, loss of access to our email has become a life-altering event with significant consequences. Such a loss may result due to a compromised password, or a judgement by the provider. 

Additionally, should we now want to migrate email providers for personal reasons, it has become an insurmountable task. Users may find themselves effectively powerless to hold their service providers accountable.

\subsubsection{Spam}

While email was envisioned as a person-to-person and machine-to-machine messaging system, the vast majority of emails are now machine-to-human. These messages largely consist of useless advertisements, but require processing power and human attention to evaluate and identify for deletion. Large-scale centralized email providers (e.g. Gmail\footnote{\url{https://mail.google.com}} and Hotmail\footnote{\url{https://outlook.live.com}}) benefit from their message volume by being able to identify similar messages sent to a large number of different customers and filter them out.

Ironically, companies we actively do business with are sending more ``opt-in'' marketing messages. These are largely served unfiltered by these centralized platforms despite the original promise of these platforms to provide users with only high-value communications. Email is now only a historical record for our business notifications.

As a result, much of our high-value communications with personal acquaintances have moved to other digital platforms like SMS, Telegram, Messenger, WhatsApp, Twitter, and Signal. However, in order to dissuade spam on these platforms, they require providing a telephone number, or email, or both, in order to have an account.

If an account starts producing too much unwanted content on these systems, the account is restricted or deleted. The pinned phone number, or email address, is permanently banned. These identifiers are difficult to obtain due to the restrictions around sign-ups placed by email providers, and the cost associated with purchasing a phone number.

\subsubsection{Consequences}

While large services have provided us with much-needed convenience, and enabled less-technical individuals to participate in the Internet, centralized providers must also generate revenue to maintain their infrastructure and generate a profit. As such, many of these services provide access to email, and websites, for ``free.'' However, ``if you're not paying for the product, you are the product.''

Some providers still provide paid email access while essentially offering privacy as a product. However, they still have access to the same data about their users as free services. There becomes a financial incentive to sell this data while maintaining a guise of privacy. Regardless of the ethics of paid providers, emails exchanged with free email providers (e.g. Gmail) are still indexes and categorized for advertising.

Again, in non-email systems, our identity is increasingly tied to our email or phone numbers. This association means that there is a clear association between all other accounts and digital interactions. Being able to collate all this data about a user, and form a more complete profile, is extremely valuable to advertisers.

Indeed, various companies (e.g. LiveRamp) purchase data across multiple services, and collate it based on emails and phone numbers. Device fingerprinting is also employed to combine this data with ones web browsing history. Device fingerprints also provide a way to associate email addresses and phone numbers should users log in to two different accounts during the same session or on the same computer.

The stated purpose of this is to provide highly-specific advertising. This may be desirable to users in finding products they want. However, are many other purposes this data is used for which are of concern. These other uses are outside the scope of this paper.

\section{The CashWeb Protocol}

\subsection{Philosophy}

In order to provide an alternative to the existing system we adhere to the following principles:

\subsubsection{Simplicity}

In order to support the needs of non-technical users, CashWeb must be a solution which requires minimal technical expertise. It must be possible to use third-party providers which provide hosting, and operation, of various services.

\subsubsection{Migratability}

Users of the CashWeb system must have control over access to their own identities, and the ability to migrate from one service provider to another. This enables them to hold service providers accountable for their actions. This requires a name resolution system which is distributed in nature. A functional name resolution system must be resistant to denial of service attacks, and have no central authority which can censor particular identities.

\subsubsection{Recoverability}

In the case of a loss of identity, it must be possible to recover gracefully. In order to make authentication transferable from one service provider to another, asymmetric cryptography must be employed so that identification is not a responsibility of CashWeb service providers.

\subsubsection{Security \& Privacy}

In order to protect the privacy of users, the contents of messages must not be readable by third parties, including CashWeb service providers. All communications between two parties should be encrypted by default. The systems must be compatible with existing overlay networks to provide additional security such as Tor.

\subsubsection{Permissionlessness}

While individual software implementation may be privately maintained, the protocol must offer a low barrier of entry into the ecosystem. Reference implementations should be provided.

\subsection{Central Concepts}

In order to meet the above requirements for the CashWeb system, the following design decisions were made.

\subsubsection{Web Standards}

The CashWeb system adheres to the established web standards to allow quick and easy integration into existing protocols and infrastructure. "Bearer"-style tokens are used extensively in combination with existing cryptocurrency payment standards to allow authenticated access to resources to be purchased with Bitcoin tokens.

\subsubsection{Cryptocurrency}

In order to maintain secure communication, and the permissionless nature of the CashWeb system, it must be impractical for a single party to send large volumes of unsolicited messages. All messages sent should impose a cost on the sender that is paid to the recipient. In order to support this requirement, payments need to be at the center of the design.

Traditional systems require trusted third parties and complex integration's with the traditional banking system. Visionaries like Hal Finney conceived of this problem being solved through "reusable proof of work" (RPoW). However, his original design was impractical, but ultimately made possible through the use of Bitcoin Tokens.

Unfortunately, the Bitcoin network does not support the transaction volume which would be associated with a widely used messaging system. Most other cryptocurrency systems also do not intend to support volumes on the order of email (sans spam). The ones that do support these volumes have centrally managed economic policies. Such management would give them authority over the ability to send and receive messages.

Thus, Bitcoin Cash was selected due to its roadmap being highly compatible with the requirements of the CashWeb project. The roadmap purports to desire to support the majority of payments on earth. Also, It uses a proof-of-work with no central issuing authority for tokens issuance.

Additionally, using a cryptocurrency, instead of traditional banking integrations, synergizes well with the secure communication requirement. The same keys used to send and receive funds can also be used to provide encryption for messages.

\subsubsection{Identity}

Each user identity is pseudonymous, and associated with a public key. These public identities can be easily and inexpensively generated from a single master key. Each identity key is acknowledged by the network via various small payments to the miners of the Bitcoin Cash network. These small payments include a cryptographically verifiable commitment to the identity which can provide proof that the payment was made to any third party.

Additionally, these pseudonyms can be made such that they can be proven to have been derived from another hidden key at a later date. Such proofs enable the specific key associated with a pseudonym to be revoked, and rotated, in a trustless way. This enables the contacts of the pseudonym to be informed without the need for re-establishing trust. The specific details of these identity schemes are left to further detailed protocol specifications, and the protocol is extensible to future schemes should a need arise.

\subsubsection{Message Format}

For all messages within the CashWeb system use the Protocol Buffer message format. "Protobufs" are now in wide use, easy to implement in a variety of languages, and serializable to binary.

\subsection{Components}

\subsubsection{Keyservers}

The CashWeb protocol includes a network of keyservers which provide a public \& distributed metadata registry. The registry is intended to track small amounts of metadata associated with cryptographic keys. Each entry in the keyservers is replicated across the network to provide censorship resistance. A peer-to-peer protocol is included which provides eventual consistency.

This metadata is indexed by the hash of ones public key and includes said public key, a body of information, and a signature covering the body providing integrity, authentication, and non-repudiation. Metadata updates are permissioned by providing valid signatures.

Uploading data to the keyserver is protected by a "Proof-of-Payment protocol" (POP protocol). This provides a way to anchor on-chain value to specific updates and therefore allowing DoS resistant replication across the keyserver network.

The specialised CashWeb keyserver has the following benefits over existing GPG infrastructure:
\itemize{
  \item Anti-DDoS mechanisms are considered from inception and hence we can arrive at a more simple and robust overall design.
  \item HTTP2 makes it significantly simpler to interact with, for example, it is immediately compatible with off-the-shelf load balancers.
  \item The payload format is more concise than what existing keyservers provide. However, X.509 certificates can also be provided inside an entry associated with a given address.
}

The CashWeb keyserver can be used for a wide-range of applications (which are eluded to below), however our primary use-case is to record a pointer to the specific relay server managing that users messages. In this way any user with access to the keyserver network and a hosted address, may lookup the address on the keyserver network and then redirect to their specific relay server in order to bootstrap communication. 

Another function of keyservers is to provide revocations of keys in the event that a user loses their on-line private key. The keyservers enable the publication of new keys to existing contacts in a trustless manner. This allows for key-rotation on Bitcoin, which has been a significant deficiency in all cryptocurrencies since the Bitcoin whitepaper was written.

\subsubsection{Relay Servers}

Relay servers provide the combined purpose of both POP and SMTP servers. They accept messages on behalf of the clients, and verify basic integrity of these messages. They also host profile information including avatars and other information. While this server currently only provide messaging, profile names, and icons, they can easily be extended to provide status messages, microblogs, and other potentially useful functions.

The distinction here between keyservers and relay servers is made due to seperation of concerns:
* Keyservers provide global replication and therefore censorship-resistance for small amounts of unencrypted data.
* Relay servers serve only specific users and therefore can cheaply host large amounts of encrypted personal data.

Uploading to, and pulling messages from, the relay server should be protected by the Proof-of-Payment protocol. By using a standardized authentication and authorization mechanism, along with the global keyserver network, users can easily migrate between relay service providers should they choose.

\subsubsection{Messaging Client}

An end-user client is necessary for interacting with this system in a easy-to-use way. The client serves to manage funds associated with sending and receiving messages, update keyservers as to which relay server the user accepts messages, and connect to and process messages received from the relay servers.

The messaging clients are the most complex portion of the CashWeb system. The majority of the protocol is handled via encrypted structured payloads that wallets needs to parse and process. Both the keyserver and the relay server are for the most part ignorant as to the protocols that wallets may use to talk to each other.

This allows the functionality to evolve gracefully over time without needing major changes to the underlying infrastructure. Wallet developers may add and evolve functionality over time. Wallets need only ignore payload types they don't understand, while enabling protocol changes without needing wide-scale consensus on additions.

\subsubsection{Protocol Flow}

\begin{figure}[H]
\centering
\scalebox{.75}{
    \begin{msc}[c]{Profile Retrieval}
      \declinst{A}{}{ Client }
      \declinst{B}{}{ Keyserver }
      \declinst{C}{}{ Relay Server }
      \regionstart{activation}{A}
      \regionstart{activation}{B}
      \mess{GET /keys/\{address\}}{A}{B}
      \nextlevel[2]
      \mess{Metadata}{B}{A}
      \regionend{B}
      \nextlevel[1]
      \action{Extract Relay IP}{A}
      \nextlevel[2]
      \regionstart{activation}{C}
      \mess{GET /profiles/\{address\}}{A}[0.75]{C}
      \nextlevel[2]
      \mess{Profile}{C}[0.75]{A}
      \regionend{A}
      \regionend{C}
      \nextlevel[1]
    \end{msc}
    }
\end{figure}

\begin{figure}[H]
\centering
\scalebox{.75}{
    \begin{msc}[c]{Message Send}
     \declinst{A}{ }{ Client }
     \declinst{B}{}{ Keyserver }
     \declinst{C}{}{ Relay Server }
     \declinst{D}{}{ Bitcoin Node }
     \regionstart{activation}{A}
     \regionstart{activation}{B}
     \mess{GET /keys/\{address\}}{A}{B}
     \nextlevel[2]
     \mess{AddressMetadata}{B}{A}
     \regionend{B}
     \nextlevel[1]
     \action{Extract Relay IP}{A}
     \nextlevel[2]
     \regionstart{activation}{C}
     \mess{PUT /messages/\{address\}}{A}[0.75]{C}
     \nextlevel[1]
     \regionstart{activation}{D}
     \mess{Stamp Transactions}{C}{D}
     \nextlevel[2]
     \mess{Accepted}{D}{C}
     \regionend{D}
     \nextlevel[1]
     \mess{Ok}{C}[0.75]{A}
     \regionend{A}
     \regionend{C}
     \nextlevel[1]
    \end{msc}
}
\end{figure}

\section{Applications}

\subsection{Standardized Authorization \& Authentication}

The Proof-Of-Payment (POP) protocol enables the standardized. and seemless, use of HTTP APIs as traditionally used in the technology industry, but without the use of complicated billing infrastructure and account management. It allows the purchasing pseudnonymously of a JWT API token, without requiring accounts, billing frontends, or other complicated infrastructure.

\subsection{Distributed Identity Management}

Having a standard keyserver infrastructure benefits a range of applications. Such as cryptographically secure, but updatable, contact exchange via QR codes or other mediums. The ability to rotate this information allows for important key revocation, and rotation events, for end-users. It provides a comprehensive mechanism for managing online identities in a decentralized and trustless manner. Having neutral indetity infrastructure provides a strong incentive for participation from all online users; unlike centralized identity providers (e.g. Google, Facebook, etc.)

\subsection{Open Messaging}

The combination of the POP protocol, keyservers, and relay servers, allows for advanced privacy features and SPAM-free communication. The ability to send structured messages with attached value, allows for all kinds of fee-based human-to-human, human-to-machine, machine-to-human, and machine-to-machine message processing. The most obvious application of this is peer to peer payments and messaging. However, the potential for other interesting services such as robots, which exchange value as well as information, can be imagined.

\section{Conclusion}

The CashWeb protocols aim to provide censorship-resistant solutions to common technological problems which continually are reinvented in proprietary ways. CashWeb, like the underlying cryptocurrency technology, allows for the disintermediation of authentication, identity management, and messaging.

Attaching payments to P2P communications enables the disruption of existing internet power structures. Centralized moderation is obviated, and thus CashWeb provides digital ``neutral ground'' for the collaboration of all willing parties. It removes the incentive for continued "walled-garden" communications networks.

Digital currency combined with messaging, as originally imagined by Hal Finney\cite{finney2004rpow}, provides equal footing for all participants in the global dialog. This is a critical step, and critical infrastructure, in maintaining human rights as technology continues to evolve. It also has the potential to change the way humans communicate and think by enabling us to focus our attention on information which is truly valuable.
\cite{nakamoto2008bpp}

\bibliographystyle{unsrt}
\bibliography{citations}

\end{document}
